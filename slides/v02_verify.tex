\input{slides_common}

\newif\ifbook
\input{../shared/chisel}

\title{Verification of Digital Designs: Week 2}
\author{Martin Schoeberl}
\date{\today}
\institute{Technical University of Denmark\\
Embedded Systems Engineering}

\begin{document}

\begin{frame}
\titlepage
\end{frame}

\begin{frame}[fragile]{Overview}
\begin{itemize}
\item More on tooling
\item Some more Scala (+ repetition)
\item An example test of an ALU
\item Regression tests and continuous integration
\item Presentation of reading material
\item Lab: test an accumulator (from the Leros processor)
\end{itemize}
\end{frame}

\begin{frame}[fragile]{Tools: git and GitHub}
\begin{itemize}
\item git is a distributed version-control system
\item You \emph{may} also use GitHub as code host
\item \code{git clone path} local copy from a server
\item \code{git pull} get updates
\item \code{git commit -a -m "message"} add your changes locallye
\item \code{git push} push to the server
\item \code{git add filename} add a file to the repo
\item Always start with git pull
\item Concurrent edits are merged on a line-by-line bases
\item When sharing text, have manual line breaks to minimize chances of conflict
\end{itemize}
\end{frame}

\begin{frame}[fragile]{Tools: make}
\begin{itemize}
\item \code{make} is for automation
\item A list a dependencies and commands to run
\item Can become complex
\item Keep it simple just to remember commands
\item Look into chisel-example
\end{itemize}
\end{frame}

\begin{frame}[fragile]{Using make}
\begin{itemize}
\item Run it with:
\begin{itemize}
\item \code{make}
\end{itemize}
\item Run a target:
\begin{itemize}
\item \code{make target}
\end{itemize}
\item Targets are described in a \code{Makefile}
\begin{itemize}
\item \code{target: dependency}
\item \code{[tab] command}
\end{itemize}
\item Let us explore it now
\item Install \code{make} and write a simple \code{Makefile}
\end{itemize}
\end{frame}

\begin{frame}[fragile]{Scala Values and Variables}
\begin{chisel}
// A value is a constant
val i = 0
// No new assignment; this will not compile
i = 3

// A variable can change the value
var v = "Hello"
v = "Hello World"

// Type usually inferred, but can be declared
var s: String = "abc"
\end{chisel}
\end{frame}

\begin{frame}[fragile]{Simple Loops}
\begin{chisel}
// Loops from 0 to 9
// Automatically creates loop value i
for (i <- 0 until 10) {
  println(i)
}
\end{chisel}
\end{frame}

\begin{frame}[fragile]{Scala Classes}
\begin{chisel}
// A simple class
class Example {
  // A field, initialized in the constructor
  var n = 0
  
  // A setter method
  def set(v: Int) = {
    n = v
  }
  
  // Another method
  def print() = {
    println(n)
  }
}
\end{chisel}
\end{frame}

\begin{frame}[fragile]{Functions}
\begin{itemize}
\begin{chisel}
class Example {
  
  def compute(a: Int, b: Int): Int = {
    a + b
  }
}
\end{chisel}
\item Last expression is the return value
\end{itemize}
\end{frame}

\begin{frame}[fragile]{Functions}
\begin{itemize}
\begin{chisel}
class Example {
  
  def complexCompute(a: Int, b: Int, c: Int): Int = {
  
    def add(x: Int, y: Int) {
      x + y
    }
    
    add(a, b) + c
  }
}
\end{chisel}
\item We can define local functions
\item To better organize our code
\end{itemize}
\end{frame}

\begin{frame}[fragile]{Functions}
\begin{itemize}
\begin{chisel}
class Example {
  
  def complexCompute(a: Int, b: Int, c: Int): Int = {
  
    def add() {
      a + b
    }
    
    add() + c
  }
}
\end{chisel}
\item Local functions have access to outer variables
\end{itemize}
\end{frame}


\begin{frame}[fragile]{Regression Tests}
\begin{itemize}
\item Tests are collected over time
\item When a bug is found, a test is written to reproduce this bug
\item Collection of tests increases
\item Runs every night to test for \emph{regression}
\begin{itemize}
\item Did a code change introduce a bug in the current code base?
\end{itemize}
\end{itemize}
\end{frame}


\begin{frame}[fragile]{Continuous Integration (CI)}
\begin{itemize}
\item Next logical step from regression tests
\item Run all tests whenever code is changed
\item Automate this with a repository, e.g., on GitHub
\item Run CI on Travis (with GitHub integration)
\item Show about this on the Chisel book
\begin{itemize}
\item Show \code{sbt test}
\item Mails from travis
\item Live demo on travis
\end{itemize}
\item \url{https://travis-ci.com/schoeberl/chisel-book}
\end{itemize}
\end{frame}

\begin{frame}[fragile]{Lab Time I: Write a ScalaTest}
\begin{itemize}
\item Setup a Scala project with \code{build.sbt}
\item ScalaTest: write a test suit to test \code{Int} that 2 + 3 = 5, and two more integer tests
\item Maybe add a \code{Makefile}
\end{itemize}
\end{frame}

\begin{frame}[fragile]{Lab Time II: Test the accumulator}
\begin{itemize}
\item ALU + register
\item Exhaustive testing is not an option
\item Corner cases plus random
\end{itemize}
\end{frame}

\begin{frame}[fragile]{Lab Time III: CI with Travis}
\begin{itemize}
\item Create a repo on GitHub
\item Add the ALU plus your tester
\item Connect it with Travis
\item you need a \code{.travis.yml}
\item Look into the Chisel book source for an example
\end{itemize}
\end{frame}
%\begin{frame}[fragile]{Summary}
%\begin{itemize}
%\item xxx
%\end{itemize}
%\end{frame}

\end{document}

\begin{frame}[fragile]{Title}
\begin{itemize}
\item abc
\end{itemize}
\end{frame}
