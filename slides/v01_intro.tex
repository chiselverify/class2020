\input{slides_common}

\newif\ifbook
\input{../shared/chisel}

\title{Verification of Digital Designs: Introduction}
\author{Martin Schoeberl}
\date{\today}
\institute{Technical University of Denmark\\
Embedded Systems Engineering}

\begin{document}

\begin{frame}
\titlepage
\end{frame}


\begin{frame}[fragile]{Overview}
\begin{itemize}
\item Motivation
\item Course organization
\item Languages for digital hardware design
%\item Testing (see 2.1.4)
\item Debugging, Testing, and Verification
\item A little bit of Scala
\item An Exercise
\end{itemize}
\end{frame}


\begin{frame}[fragile]{Motivation}
\begin{itemize}
\item We had a meeting with DK industry in June
\item Missing design verification
\item For one developer there are 2--3 verification engineers
\item Testing is considered boring, but it does not have to be
\item Best is to be a developer and a verification engineer
\item Change roles, we will do in this course
\end{itemize}
\end{frame}

\begin{frame}[fragile]{Course Organization}
\begin{itemize}
\item This is a special course
\item Not just me giving talks and preparing exercises
\begin{itemize}
\item But I will bring in some
\item You will bring in some as well
\end{itemize}
\item This is a lot about self study
\item You will bring up material
\item We will use GitHub for material, exercises, project
\begin{itemize}
\item Slides are there as well
\item \url{https://github.com/chisel-uvm/class2020}
\item Let's signup right now ;-)
\end{itemize}
\end{itemize}
\end{frame}

\begin{frame}[fragile]{Technicalities}
\begin{itemize}
\item We will use Chisel/Scala for tesing
\item VHDL, SystemVerilog, UVM are optional
\item You need to setup your laptop
\begin{itemize}
\item see: https://github.com/schoeberl/chisel-lab/blob/master/Setup.md
\item We will not use an FPGA
\end{itemize}
\end{itemize}
\end{frame}

\begin{frame}[fragile]{Reading Material}
\begin{itemize}
\item No \emph{good} book on DV
\item We need to find
\begin{itemize}
\item Web sites
\item Blogs
\item Articles (popular, e.g., EE Times)
\item Paper (conferences)
\item Look what software people do
\item ...
\end{itemize}
\item Your homework: search literature till next week
\item Present what you found
\end{itemize}
\end{frame}


\begin{frame}[fragile]{This is an Open-Access/Open-Source Course}
\begin{itemize}
\item Almost all material is public visible
\item Slides are open access
\item Lab material is open access
\item Hosted on GitHub
\begin{itemize}
\item \textbf{You} can contribute with a pull request
\item Becoming an author of this course :-)
\end{itemize}
\item The Chisel book is freely available
\end{itemize}
\end{frame}



\begin{frame}[fragile]{Chisel Overview}
\begin{itemize}
\item A hardware \emph{construction} language
\begin{itemize}
\item Constructing Hardware In a Scala Embedded Language
\item If it compiles, it is synthesisable hardware 
\item Say goodby to your unintended latches
\end{itemize}
\item Chisel is not a high-level synthesis language
\item Single source for two targets
\begin{itemize}
\item Cycle accurate simulation (testing)
\item Verilog for synthesis
\end{itemize}
\item Embedded in Scala
\begin{itemize}
\item Full power of Scala available
\item But to start with, no Scala knowledge needed
\end{itemize}
\item Developed at UC Berkeley
\end{itemize}
\end{frame}


\begin{frame}[fragile]{Chisel vs. Scala}
\begin{itemize}
\item A Chisel hardware description is a Scala program
\item Chisel is a Scala library
\item When the program is executed it generates hardware
\item Chisel is a so-called \emph{embedded domain-specific language}
\end{itemize}
\end{frame}


\begin{frame}[fragile]{Free Tools for Chisel and FPGA Design}
\begin{itemize}
\item \href{https://adoptopenjdk.net/}{Java OpenJDK 8} already installed for Java course
\item \href{https://www.scala-sbt.org/}{sbt, the Scala (and Java) build tool}
\item \href{https://www.jetbrains.com/idea/download/}{IntelliJ (the free Community version)}
\item \href{http://gtkwave.sourceforge.net/}{GTKWave}
\item \href{https://www.xilinx.com/products/design-tools/vivado/vivado-webpack.html}{Vivado WebPACK} already installed from DE1
% \item \href{http://www.altera.com/products/software/quartus-ii/web-edition/qts-we-index.html}{Quartus}
\item Nice to have:
\begin{itemize}
\item make, git
\end{itemize}
\end{itemize}
\end{frame}

\begin{frame}[fragile]{Tool Setup for Different OSs}
\begin{itemize}
\item Windows
\begin{itemize}
\item Use the installers from the websites
\end{itemize}
\item macOS
\begin{itemize}
\item \code{brew install sbt}
\item For the rest, use the installer from the websites
\end{itemize}
\item Linux/Ubuntu
\begin{itemize}
\item \code{sudo apt install openjdk-8-jdk git make gtkwave}
\item Install sbt, see \url{https://github.com/schoeberl/chisel-lab/blob/master/Setup.md}
\item IntelliJ as from the website
\end{itemize}
\end{itemize}
\end{frame}

\begin{frame}[fragile]{An IDE for Chisel}
\begin{itemize}
\item IntelliJ
\item Scala plugin
\item For IntelliJ: File - New - Project from Existing Sources..., open build.sbt
%\item But you are not compiling with Eclipse\\ and against the Chisel source
\item Show it (down to the Basys3)
\end{itemize}
\end{frame}


\begin{frame}[fragile]{A Chisel Book}
\begin{figure}
    \centering
    \href{https://github.com/schoeberl/chisel-book}{\includegraphics[scale=0.4]{../cover-small}}
\end{figure}

\begin{itemize}
\item Available in open access (as PDF)
\begin{itemize}
\item Optimized for reading on a tablet (size, hyper links)
\end{itemize}
\item Amazon can do the printout
\end{itemize}
\end{frame}

\begin{frame}[fragile]{Further Information}
\begin{itemize}
\item \url{https://www.chisel-lang.org/}
\item \url{https://github.com/freechipsproject/chisel-cheatsheet/releases/latest/download/chisel_cheatsheet.pdf}
\item \url{https://github.com/ucb-bar/chisel-tutorial}
\item \url{https://github.com/ucb-bar/generator-bootcamp}
%\item Chisel 2 documentation at \url{https://github.com/schoeberl/chisel2-doc}
%\begin{itemize}
%\item Chisel 2.2 Tutorial
%\item Getting Started with Chisel
%\end{itemize}
\item \url{http://groups.google.com/group/chisel-users}
\item \url{https://github.com/schoeberl/chisel-book}
\end{itemize}
\end{frame}


\begin{frame}[fragile]{Testing and Debugging}
\begin{itemize}
\item Nobody writes perfect code ;-)
\item We need a method to improve the code
\item In Java we can simply print the result:
\begin{itemize}
\item \code{println("42");}
\end{itemize}
\item What can we do in hardware?
\begin{itemize}
\item Describe the whole circuit and hope it works?
\item We can switch an LED on or off
\end{itemize}
\item We need some tools for \href{https://en.wikipedia.org/wiki/Debugging#/media/File:H96566k.jpg}{debugging}
\item Writing testers in Chisel
\end{itemize}
\end{frame}

\begin{frame}[fragile]{Testing with Chisel}
\begin{itemize}
\item Set input values with \code{poke}
\item Advance the simulation with \code{step}
\item Read the output values with \code{peek}
\item Compare the values with \code{expect}
\item Import following packages:
\shortlist{../code/test_import.txt}
\end{itemize}
\end{frame}

\begin{frame}[fragile]{Using \code{peek}, \code{poke}, and \code{expect}}
\begin{chisel}
// Set input values
poke(dut.io.a, 3)
poke(dut.io.b, 4)
// Execute one iteration
step(1)
// Print the result
val res = peek(dut.io.result)
println(res)

// Or compare against expected value
expect(dut.io.result, 7)
\end{chisel}
\end{frame}

\begin{frame}[fragile]{A Chisel Tester}
\begin{itemize}
\item Extends class \code{PeekPokeTester}
\item Has the device-under test (DUT) as parameter
\item Testing code can use all features of Scala
\end{itemize}
\begin{chisel}
class CounterTester(dut: Counter) extends PeekPokeTester(dut) {

  // Here comes the Chisel/Scala code
  // for the testing
}
\end{chisel}
\end{frame}

\begin{frame}[fragile]{Example DUT}
\begin{itemize}
\item A device-under test (DUT)
\shortlist{../code/test_dut.txt}
\end{itemize}
\end{frame}

\begin{frame}[fragile]{A Simple Tester}
\begin{itemize}
\item Just using \code{println} for manual inspection
\shortlist{../code/test_bench_simple.txt}
\end{itemize}
\end{frame}


\begin{frame}[fragile]{The Main Program for the Test}
\begin{itemize}
\item Extend an App and invoke the \code{iotesters} driver
\item With the DUT and the tester
\shortlist{../code/test_main_simple.txt}
\end{itemize}
\end{frame}

\begin{frame}[fragile]{A Real Tester}
\begin{itemize}
\item Poke values and \code{expect} some output
\shortlist{../code/test_bench.txt}
\end{itemize}
\end{frame}

\begin{frame}[fragile]{ScalaTest}
\begin{itemize}
\item Testing framework for Scala
\item \code{sbt} understands ScalaTest
\item Run all tests:
\code{sbt test}
\item When all \code{expect}s are ok, the test passes
\item A little bit funny syntax
\item Add library to \code{build.sbt}
\begin{chisel}
libraryDependencies += "org.scalatest" %% "scalatest" % "3.0.5" % "test"
\end{chisel}
\item Import ScalaTest library
\begin{chisel}
import org.scalatest._
\end{chisel}
\end{itemize}
\end{frame}

\begin{frame}[fragile]{ScalaTest Version of our Tester}
\shortlist{../code/scalatest_simple.txt}
\end{frame}

\begin{frame}[fragile]{Generating Waveforms}
\begin{itemize}
\item Waveforms are timing diagrams
\item Good to see many parallel signals and registers
\item Additional parameters: \code{"--generate-vcd-output", "on"}
\item IO signals and registers are dumped
\item Option \code{--debug} puts all wires into the dump
\item Generates a .vcd file
\item Viewing with GTKWave or ModelSim
\end{itemize}
\end{frame}


\begin{frame}[fragile]{Call the Tester}
\begin{itemize}
\item Using here ScalaTest
\item Note \code{Driver.execute}
\item Note \code{Array("--generate-vcd-output", "on")}
\end{itemize}
\begin{chisel}
class Count6WaveSpec extends
  FlatSpec with Matchers {
  
  "CountWave6 " should "pass" in {
    chisel3.iotesters.Driver.
    execute(Array("--generate-vcd-output", "on"),() => new Count6)
    { c => new Count6Wave(c) }
    should be (true)
  }
}
\end{chisel}
\end{frame}


\begin{frame}[fragile]{Test Driven Development (TDD)}
\begin{itemize}
\item Software development process
\begin{itemize}
\item Can we learn from SW development for HW design?
\end{itemize}
\item Writing the test first, then the implementation
\item Started with extreme programming
\begin{itemize}
\item Frequent releases
\item Accept change as part of the development
\end{itemize}
\item Not used in its pour form
\begin{itemize}
\item Writing all those tests is simply considerer too much work
\end{itemize}
\end{itemize}
\end{frame}


\begin{frame}[fragile]{Testing versus Debugging}
\begin{itemize}
\item Debugging is during code development
\item Waveform and println are easy tools for debugging
\item Debugging does not help for regression tests
\item Write small test cases for regression tests
\item Keeps your code base \emph{intact} when doing changes
\item Better confidence in changes not introducing new bugs
\end{itemize}
\end{frame}

\begin{frame}[fragile]{Scala}
\begin{itemize}
\item Is object oriented
\item Is functional
\item Strongly typed with very good type inference
\item Runs on the Java virtual machine
\item Can call Java libraries
\item Consider it as Java++
\begin{itemize}
\item Can almost be written like Java
\item With a more lightweight syntax
\item Compiled to the JVM
\item Good Java interoperability
\item Many libraries available
\end{itemize}
\item \url{https://docs.scala-lang.org/tour/tour-of-scala.html}
\end{itemize}
\end{frame}


\begin{frame}[fragile]{Scala Hello World}
\begin{chisel}
object HelloWorld extends App {
  println("Hello, World!")
}
\end{chisel}
\begin{itemize}
\item Compile with \code{scalac} and run with \code{scala}
\item You can even use Scala as scripting language
\item Show both
\item Scala has a REPL, show it
\end{itemize}
\end{frame}

\begin{frame}[fragile]{Scala Values and Variables}
\begin{chisel}
// A value is a constant
val i = 0
// No new assignment; this will not compile
i = 3

// A variable can change the value
var v = "Hello"
v = "Hello World"

// Type usually inferred, but can be declared
var s: String = "abc"
\end{chisel}
\end{frame}

\begin{frame}[fragile]{Simple Loops}
\begin{chisel}
// Loops from 0 to 9
// Automatically creates loop value i
for (i <- 0 until 10) {
  println(i)
}
\end{chisel}
\end{frame}

\begin{frame}[fragile]{Conditions}
\begin{chisel}
for (i <- 0 until 10) {
  if (i%2 == 0) {
    println(i + " is even")
  } else {
    println(i + " is odd")
  }
}
\end{chisel}
\end{frame}

\begin{frame}[fragile]{Scala Arrays and Lists}
\begin{chisel}
// An integer array with 10 elements
val numbers = new Array[Integer](10)
for (i <- 0 until numbers.length) {
  numbers(i) = i*10
}
println(numbers(9))


// List of integers
val list = List(1, 2, 3)
println(list)
// Different form of list construction
val listenum = 'a' :: 'b' :: 'c' :: Nil
println(listenum)
\end{chisel}
\end{frame}


\begin{frame}[fragile]{Scala Classes}
\begin{chisel}
// A simple class
class Example {
  // A field, initialized in the constructor
  var n = 0
  
  // A setter method
  def set(v: Integer) = {
    n = v
  }
  
  // Another method
  def print() = {
    println(n)
  }
}
\end{chisel}
\end{frame}

\begin{frame}[fragile]{Scala (Singleton) Object}
\begin{chisel}
object Example {}
\end{chisel}
\begin{itemize}
\item For \emph{static} fields and methods
\begin{itemize}
\item Scala has no static fields or methods like Java
\end{itemize}
\item Needed for \code{main}
\item Useful for helper functions
\end{itemize}
\end{frame}

\begin{frame}[fragile]{Singleton Object for the \code{main}}
\begin{chisel}
// A singleton object
object Example {
  
  // The start of a Scala program
  def main(args: Array[String]): Unit = {
    
    val e = new Example()
    e.print()
    e.set(42)
    e.print()
  }
}
\end{chisel}
\begin{itemize}
\item Compile and run it with sbt (or within Eclipse/IntelliJ):
\end{itemize}
\begin{chisel}
sbt "runMain Example"
\end{chisel}
\end{frame}

\begin{frame}[fragile]{Scala Build Tool (sbt)}
\begin{itemize}
\item Downloads Scala compiler if needed
\item Downloads dependent libraries (e.g., Chisel)
\item Compiles Scala programs
\item Executes Scala programs
\item Does a lot of magic, maybe too much
\item Compile and run with:
\end{itemize}
\begin{chisel}
sbt "runMain simple.Example"
\end{chisel}
\begin{itemize}
\item Or even just:
\end{itemize}
\begin{chisel}
sbt run
\end{chisel}
\end{frame}

\begin{frame}[fragile]{Build Configuration}
\begin{itemize}
\item Defines needed Scala version
\item Library dependencies
\item File name: \code{build.sbt}
\end{itemize}
\begin{chisel}
scalaVersion := "2.11.7"

resolvers ++= Seq(
  Resolver.sonatypeRepo("snapshots"),
  Resolver.sonatypeRepo("releases")
)

libraryDependencies += "edu.berkeley.cs" %% "chisel3" % "3.1.2"
libraryDependencies += "edu.berkeley.cs" %% "chisel-iotesters" % "1.2.2"
\end{chisel}
\end{frame}

\begin{frame}[fragile]{File Organization in Scala/Chisel}
\begin{itemize}
\item A Scala file can contain several classes (and objects)
\item For large classes use one file per class with the class name
\item Scala has packages, like Java
\item Use folders with the package names for file organization
\item \code{sbt} looks into current folder and \code{src/main/scala/}
\item Tests shall be in \code{src/test/scala/}
\end{itemize}
\end{frame}

\begin{frame}[fragile]{File Organization in Scala/Chisel}
\dirtree{%
.1 project.
.2 src.
.3 main.
.4 scala.
.5 package.
.6 sub-package.
.3 test.
.4 scala.
.5 package.
.2 target.
.2 generated.
}
\end{frame}

\begin{frame}[fragile]{Chisel in Scala}
\begin{itemize}
\item Chisel components are Scala classes
\item Chisel code is in the constructor
\item Executed at object creation time
\item Builds the network of hardware objects
\item Testers are written in Scala to drive the tests
\item You can write a reference simulation in Scala and compare with Chisel
\end{itemize}
\end{frame}

\begin{frame}[fragile]{Summary}
\begin{itemize}
\item This is a special course
\item We will work together to learn about testing and verification
\item You will present reading material next week
\item We meet again next Tuesday 13:00 in 322/123
\end{itemize}
\end{frame}

\begin{frame}[fragile]{Lab Time: Hello World Testing}
\begin{itemize}
\item Write test/verification code for a 5:1 multiplexer
\item Project and DUT in GitHub lab1
\item \url{https://github.com/chisel-uvm/class2020}
\end{itemize}
\end{frame}



\end{document}

\begin{frame}[fragile]{Title}
\begin{itemize}
\item abc
\item abc
\item abc
\item abc
\item abc
\item abc
\item abc
\item abc
\item abc
\end{itemize}
\end{frame}

